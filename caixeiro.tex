\documentclass[a4paper, 12pt]{article} %papel,  tamanho da fonte, tipo do documento
\usepackage{times}  %fonte
\usepackage[top = 3cm, bottom = 2cm, left = 3cm, right = 2 cm]{geometry} %margem
\usepackage[utf8]{inputenc} %acentos e etc
\usepackage{amsmath, amsfonts, amssymb} %simbolos
\usepackage{float} %forçar posição imagens
\usepackage{colortbl} %%mudar cor da linha da tabela: usar comando \rowcolor[rgb]{0, 0, 0}
\definecolor{lightgray}{gray}{0.9}
\usepackage{fixltx2e} % subscrito
\usepackage{graphicx} %inserir imagens
\usepackage[portuguese]{babel} %colocar em português
\usepackage[bf, BF]{subfigure} %figuras lado a lado
\usepackage{indentfirst}  %margem da primeira linha
\setlength{\parindent}{42.52 pt} % tamanho identação primeira linha
\usepackage[nottoc]{tocbibind}  %referências no sumario
\usepackage{titlesec} %editar titulos das seções
\usepackage{setspace} %espaçamento
\usepackage[footnotesize, bf]{caption} %Tamanho da
\captionsetup{ textfont=bf}   		   %%legenda e negrito
\usepackage[table,xcdraw]{xcolor} % pra poder usar tabelas do tablesgenerator.com
\titleformat{\subsection}{\normalsize \bfseries}{}{0 pt}{\thesubsection.\quad} %configuração do título
\usepackage[portuguese, ruled, linesnumbered]{algorithm2e} %pseudocódigos
\usepackage{listings}
\lstset{
basicstyle=\ttfamily\small, 
}
\usepackage{pgfplots} %graficos
\pgfplotsset{width = 7cm, compat = 1.9}
\date{2018}
\doublespacing
\begin{document}
	\begin{titlepage}
\flushleft
	
		\begin{minipage}{1\linewidth}
	\centering
	
			UNIVERSIDADE DE SÃO PAULO\\
			INSTITUTO DE CIÊNCIAS  MATEMÁTICAS E DE COMPUTAÇÃO\\
			ESTRUTURAS DE DADOS I\\
			\end{minipage}
			\\[6.7 cm]
			\begin{center}
			ABORDAGEM DE FORÇA BRUTA PARA O PROBLEMA DO CAIXEIRO VIAJANTE\\[5 cm]
			\end{center}
			\begin{flushleft}
				
				MATHEUS DE CASTRO SINISCARCHIO - \\
				MARCUS VINÍCIUS MEDEIROS PARÁ - 11031663\\
				FRANCISCO MATTOS FORTES - 4590431\\ \vspace{\fill}
			\end{flushleft}
		\begin{center}
			SÃO CARLOS/SP\\2018/01
		\end{center}
		

	\end{titlepage}
	\renewcommand{\contentsname}{\normalsize{SUMÁRIO}}
	\tableofcontents
	\titleformat{\section}{\normalsize \bfseries}{}{0 pt}{\thesection.\quad}
	
	\newpage
	
	\section{Introdução}
	
	
	Problemas de otimização são modelos matemáticos que consistem encontrar um soluções que obedeçam um conjunto de restrições e que tornem um determinada função, nesse conjunto solução, máxima ou mínima. O domínio da função objetivo é $( x_1 x_2 ... x_n)^T \in R^n$, onde $n$ é a quantidade de variáveis envolvidas; e as soluções que obedecem às restrições são chamadas de soluções factíveis.
	
	Nesse trabalho, será abordada a classe de problemas de otimização conhecida como problemas de caixeiro viajante, cuja descrição é dada na seção \ref{sec:description}. Para resolvê-lo, iremos usar a abordagem de força bruta, que não é a solução usual, por ser menos eficiente na maioria dos casos. No entanto, o objetivo principal desse trabalho é implementar uma estrutura de dados que se adeque à formulação do problema dado e permita que as operações necessárias sejam realizadas de forma eficiente. É válido notar que, mesmo quando as operações sobre os dados podem ser realizadas eficientemente, não necessariamente o problema é resolvido de forma eficiente.
		
	\section{Abordagens de força bruta}
	
	Uma das alternativas para resolver problemas de otimização é calcular o valor da função objetivo para todas as soluções factíveis e escolher a solução em que a função objetivo tenha o menor valor possível. Abordagens desse tipo, que buscam todas as soluções factíveis, são chamadas de força bruta e avaliadas como completas e ótimas. Na seção \ref{sec:modelagem}, será analisado o desempenho desse tipo de abordagem para o problema em questão.

	\section{Problemas de caixeiro viajante} \label{sec:description}
	A classe de problemas do caixeiro-viajante é um conjunto de problemas clássicos na área de otimização não-linear cujo estudo foi importante no desenvolvimento de soluções de problemas envolvendo grafos. Originalmente, envolve um caixeiro-viajante que tem o objetivo de visitar um conjunto de cidades percorrendo a menor distância possível. Três restrições são impostas: o caixeiro não pode passar pela mesma cidade duas vezes, deve visitar todas as cidades, e deve finalizar seu percurso na mesma cidade em que começou.
	
	\section{Modelagem do problema} \label{sec:modelagem}
	Para a modelagem do problema, podemos enumerar as cidades de $1$ a $n$ e definir distância para ir da cidade $i$ até a cidade $j$ como $c_{ij}$. É lógico que $c_{ij} = c_{ji}$, pois dadas duas cidades, a distância de $i$ até $j$ é a mesma de $j$ para $i$. Além disso, para simlificação do problema, iremos supor que podemos viajar de qualquer cidade para qualquer cidade, ou seja $\exists c_{ij} \geq 0, \forall i \neq j$ e $i,j \in {1, ..., n}$.
	
	Notemos que a quantidade de cidades é conhecida de antemão, e que, se conhecermos $c_{ij}$, também conhecemos $c_{ji}$. Portanto, podemos inferir que uma solução adequada pode ser sequencial dinâmica, para alocarmos em tempo de execução a memória necessária. Também, não é necessário armazernar $c_{ij}$ ao mesmo tempo que $c_{ji}$, o que permite a economia de memória. O acesso a determinada cidade é a operação mais frequente, enquanto a inserção é realizada apenas no início do problema e remoções só ocorrem ao fim.
	
	Com essas informações, podemos descrever o conjunto de cidades por uma lista sequencial, em que o i-ésimo cada elemento representa a cidade $i$. Cada elemento $i$ também contém uma lista, em que o seu j-ésimo elemento representa a distância até a cidade $j$. A figura \ref{lista_gen} ilustra esse modelo.
	
	\begin{figure}[!h]
		\centering
		\includegraphics[scale=0.45]{lista_gen.jpg}
		\caption{Representação gráfica de uma lista generalizada sequencial.}
		\label{lista_gen}
	\end{figure}
	
	Com os custos armazenados em memória, resta o problema de calcular os custos de cada uma das soluções viáveis. A abordagem utilizada consiste em gerar todas as permutações possíveis das $n$ cidades para a cidade inicial escolhida pelo usuário. Cada permutação representa uma solução viável e nos permite calcular o seu custo a partir dos custos armazenados $c_{ij}$. O menor custo e sua solução correspondente são guardados durante a execução do algoritmo e, toda vez que um novo mínimo é encontrado, a variável é atualizada.
	
	Esse modelo seria inadequado se o custo de ir da cidade $i$ para a cidade $j$ fosse diferente de ir da cidade $j$ para a cidade $i$. Além disso, ele parte da hipótese que é possível ir de qualquer cidade $i$ para qualquer cidade $j \neq i$, o que dificilmente ocorre em casos reais. Ainda é custoso remover uma cidade, que exigiria a realocação de memória da lista principal, de todas as suas sublistas e, em cada sublista deslocar os seus elementos. Assim, teriamos $O(n)$ deslocamentos em cada uma das $n - 1$ sublistas e, portanto, a remoção seria $O(n^2)$.
	
	Por outro lado, a indexação dos elementos permite o acesso de qualquer um deles a custo constante $O(1)$, que é a única operação de fato realizada após a inicialização da lista. 

	\section{Algoritmo para gerar permutações}
	
	O algoritmo utilizado para gerar as permutações é o mostrado abaixo. Em resumo, a cada iteração $i$ ele troca o primeiro elemento com o i-ésimo elemento, permuta o vetor a direita recursivamente, e troca novamente o i-ésimo elemento com o primeiro elemento novamente.
	
	\begin{algorithm}[h]
	\Entrada{$vetor$, $inferior$, $superior$}
	\textbf{Função} permuta(vetor, inferior, superior) 
	\Inicio {$superior \leftarrow tamanho$\;
		\Se{$inferior == superior$}
		{calculaCusto($vetor$)\;} 
		$i \leftarrow$ 0\;
		\textbf{Para} $i \in \{inferior,...,superior\}$ \\
			\quad troca($vetor[inferior]$, $vetor[i]$)\;
			\quad permuta($vetor$, $inferior + 1$, $superior$)\;
			\quad {5 mm} troca($vetor[inferior]$, $vetor[i]$)\;
		\textbf{fim}\;
	}
	
	\caption{Algoritmo de Permutação}
	\end{algorithm}
	
	Para calcularmos a sua complexidade, podemos notar que a cada chamada recursiva, ele realiza um loop de $n - 1$ passos, onde o $n$ é o tamanho do vetor desde a posição $inferior$ até $superior$ e é decrementado a cada iteração. E, toda vez que chega à condição de parada, é necessário calcular o custo do vetor, que é uma operação $O(n)$. Então, são realizadas $O(n!)$ chamadas recursivas e cada uma delas implica em $O(n)$ operações para calcular o custo do vetor. Portanto, o algoritmo de permutação, em conjunto com o cálculo dos custos de cada vetor é $O(n \cdot n!)$.
	
	Na prática, ainda temos que armazenar a solução com menor custo e seu respectivo custo. No pior caso, as soluções são verificadas em ordem decrescente de custo e devemos trocar o vetor armazenado sempre. A cópia de um vetor em C foi implementada copiando elemento a elemento, o que é $O(n)$. Isso torna a ordem do algoritmo realizado no fim das chamadas recursivas $O(2n) = O(n)$. A complexidade não é alterada, mas o fim da recursão leva o dobro de passos. 
	
	\subsection{Detalhes sobre a implementação}
		
		O algoritmo foi implementado em C e compilado com o gcc 7.3.0 (Ubuntu 18.04).	O arquivo principal é o "CaixeiroTeste.c". Os headers criados foram "caixeiro.h" e "combinatoria.h". Os arquivos que contém a implentação das funções utilizadas são "caixeiro.c" e "combinatoria.c". Para a compilação, foram utilizados os seguintes comandos:
		
	\begin{lstlisting}[language=bash]
  $ gcc -g -c CaixeiroTeste.c -o CaixeiroTeste.o
  $ gcc -g -c caixeiro.c -o  caixeiro.o
  $ gcc -g -c combinatoria.c -o combinatoria.o
  $ gcc CaixeiroTeste.o caixeiro.o combinatoria.o -o CaixeiroTeste
  $ gcc ./CaixeiroTeste seu_arquivo.txt
	\end{lstlisting}
		
	\section{Tempo de execução}
	
	Em testes, foi medido o tempo de execução do algoritmo utilizado para calcular o menor custo com auxílio da biblioteca \textit{time.h} . Abaixo, seguem os gráficos descrevendo a medição do tempo em função da quantidade de cidades do problema. À esquerda temos medições de 4 até 7 cidades, e à direita, de 4 até 10 cidades.
	
\begin{flushleft}	
	\begin{tikzpicture}
	
	
	\begin{axis}[
		axis lines = left,
		xlabel = $n$,
		ylabel = $t(ms)$]
		
		\addplot plot coordinates{(4,0.181) (5, 0.138) (6, 0.200) (7, 0.426) };
		
	\end{axis}
	\hskip 230pt
	\begin{axis}[
		axis lines = left,
		xlabel = $n$,
		ylabel = $t(ms)$]
		
		\addplot plot coordinates{(4,0.181) (5, 0.138) (6, 0.200) (7, 0.426) (10, 66.086)};
		
	\end{axis}
	
	\end{tikzpicture}
	
	\end{flushleft}
	
	\section{Abordagem mais eficiente}
	
	Portanto, como foi possível comprovar nesse trabalho, a abordagem de força bruta não é eficiente no caso geral. Visto que gera uma explosão combinatorial no tempo gasto para resolução do problema. O que é inviável na maioria das aplicações práticas. A fim de superar essa limitação, é comum a utilização de heurísticas que direcionem a busca das soluções factíveis, em vez da realização da busca por todas as soluções.
	
	Uma heurística bastante utilizada para a classe de problemas do caixeiro-viajante é a do vizinho mais próximo. Essa abordagem consiste em, a partir de uma cidade, selecionar como próxima cidade do caminho a que ainda não fora visitada e que estiver à menor distância. Assim, esse procedimento é avaliado como completo e, embora não ótimo - pois encontra uma solução aproximada, que pode não ser a de custo mínimo global -, interessante devido a redução significativa na complexidade computacional.
	
	Para analisarmos a sua complexidade, deve-se notar que, para visitar n cidades e voltar ao início, o caixeiro deverá tomar n-1 decisões de qual a próxima cidade a ser visitada e então voltar a cidade inicial. E, a cada tomada de decisão i, o caixeiro compara as n-i opções que ainda tem para visitar. Assim, são realizados n-1 vezes n passos na resolução do problema. O que resulta em uma complexidade computacional O(n^2).
	
\end{document}
